\documentclass[]{tufte-handout}

% ams
\usepackage{amssymb,amsmath}

\usepackage{ifxetex,ifluatex}
\usepackage{fixltx2e} % provides \textsubscript
\ifnum 0\ifxetex 1\fi\ifluatex 1\fi=0 % if pdftex
  \usepackage[T1]{fontenc}
  \usepackage[utf8]{inputenc}
\else % if luatex or xelatex
  \makeatletter
  \@ifpackageloaded{fontspec}{}{\usepackage{fontspec}}
  \makeatother
  \defaultfontfeatures{Ligatures=TeX,Scale=MatchLowercase}
  \makeatletter
  \@ifpackageloaded{soul}{
     \renewcommand\allcapsspacing[1]{{\addfontfeature{LetterSpace=15}#1}}
     \renewcommand\smallcapsspacing[1]{{\addfontfeature{LetterSpace=10}#1}}
   }{}
  \makeatother

\fi

% graphix
\usepackage{graphicx}
\setkeys{Gin}{width=\linewidth,totalheight=\textheight,keepaspectratio}

% booktabs
\usepackage{booktabs}

% url
\usepackage{url}

% hyperref
\usepackage{hyperref}

% units.
\usepackage{units}


\setcounter{secnumdepth}{-1}

% citations
\usepackage{natbib}
\bibliographystyle{plainnat}


% pandoc syntax highlighting

% table with pandoc

% multiplecol
\usepackage{multicol}

% strikeout
\usepackage[normalem]{ulem}

% morefloats
\usepackage{morefloats}


% tightlist macro required by pandoc >= 1.14
\providecommand{\tightlist}{%
  \setlength{\itemsep}{0pt}\setlength{\parskip}{0pt}}

% title / author / date
\title[Growth vs Value Funds Investment]{Exploring US Mutual Funds
Market}
\author{Zahid Asghar}
\date{2022-06-24}


\begin{document}

\maketitle




\hypertarget{introduction}{%
\section{Introduction}\label{introduction}}

Its very important study highlighting performance of growth funds vs
value funds using various statistical criteria. US mutual funds industry
is one of the most important part of global economy and studying it in
depth is always of great interest for investors, academia and policy
makers. I appreciate the hardwork and effort put by the scholars in the
paper. This study aims at suggesting whether to invest in value funds or
growth funds and conclude that one should invest in small-cap value
funds. A sample data of 65 mutual funds (further divided into 6
categories large, medium, small (growth and mutual funds)) are taken
from \textbf{Morningstar database}. Various statistical criteria are
used to evaluate the performance of value funds vs growth funds and
accordingly an investment strategies mainly based upon M\^{}2 results
has been suggested.

I have following comments which may not necessarily be in order of
importance.

\hypertarget{comments}{%
\subsection{Comments}\label{comments}}

Two or three investment strategies: growth vs value vs hybrid. As such
there is no single strategy which dominates all and there are boom and
bust cycles for each strategy. Russel1000 index for growth and value
funds indicate big cycles where the two performed poorly. 1970s to 1987
(27\% fall on single day), value funds dominated while onward till 2000,
its growth funds, 2000 till 2008/09 its value funds again and now once
again during the last 12 years or so growth funds are performing better.
Funds performance is mainly a cyclical phenomenon so evaluating
performance at a given time and advising uniform strategy to investors
is not an idea practiced anywhere. poorly or good on the basis of some
formula .

\begin{quote}
\emph{Big part of interest in it lies because of its popularity}
\end{quote}

I think reason for academicians interest is

\begin{quote}
The mutual fund industry plays an increasingly important role in US
economy over the past few decades mutual fund industry have showed
tremendous growth as more and more investors investing in US mutual
funds.
\end{quote}

\begin{itemize}
\tightlist
\item
  Questions changed from one part of the document to another. I shall
  suggest a unique precise question should be designed for having a
  precise answer. Reader remains confused throughout out the document
  \textbf{whether its comparison of techniques or comparison of
  performance of growth vs value vs hybrid investment strategies or
  devising investment strategy\ldots?}
\end{itemize}

\hypertarget{claim}{%
\subsection{Claim}\label{claim}}

\begin{quote}
This study will provide important information to investors to decide
which fund to choose by considering benefits of holding growth or value
funds.
\end{quote}

\begin{itemize}
\tightlist
\item
  I think its a bit tall claim as there is no single best strategy which
  will be useful or dominate or one may say its the behavior of the
  investor which matters. In growth funds, one is interested what will
  be future earning stream (like investing in Apple in 2007/8 and then
  annualising profit of 25\% till now was growth strategy) . Growth
  investor invest in companies like \textbf{FB, Google, Netflix,
  Amazon\ldots{}} where future growth is very high but it maybe very
  costly right now. While value investor is a \textbf{Bargain hunter}
  and tries to pick stocks which are \textbf{boring, established and
  predictable as happens in mature industries}. She loves at investing
  stocks that are priced 30/40/50 discount to their worth or intrinsic
  value.
\end{itemize}

\hypertarget{technical-jargon}{%
\subsection{Technical jargon}\label{technical-jargon}}

Story telling style is missing : one is last in techniques and becomes
difficult for reader to get the crux of the matter. Jensen alpha,
Sharpe, Treynor, M-square, and other technical jargon dominate the story
instead of growth vs value vs hybrid strategies of investments.

Is this study about comparing M-square with other measures or devising
investment strategies or something else needs to be clarified.
\textbf{Survivor bias} is defined in detail but how its taken care in
this study is not clear. Simply selecting those which survived 2005-09
throughtout this study will not ensure no survivor bias unless we have
some information about those which did not survive during the period.

\begin{itemize}
\tightlist
\item
  Is this sample reasonable to make a statement or generalize results
  for 10,000 plus US mutual funds: 65/10000 is less than 0.65\% of
  population? If homogeniety, it maybe acceptable with reservation but
  with a lot of hetrogeniety. Moreover, its a lopsided sample as sample
  is not well balanced so comparing too many companies in one case and
  too few in other categories.
\end{itemize}

\begin{quote}
Not ``I'' it is ``We\ldots{}''
\end{quote}

\begin{quote}
Not ``My study'' , it should be ``Our study\ldots{}''
\end{quote}

\begin{itemize}
\tightlist
\item
  Most of the measures are very standard and has been part of the
  literature for decades. There is need just to cite those tools instead
  of providing all basic details like that of t-test, rank correlation
  and many others.
\end{itemize}

\hypertarget{findings}{%
\subsection{Findings}\label{findings}}

\begin{quote}
One of problem defined as if expenses and management fee has any effect
on mutual fund performance is found insignificant. Mean difference is
not significant for any fund category under growth or value funds. It
indicates that expenses does not cause radical shift in performance of
mutual fund under study.
\end{quote}

-what variables and data are used for expenses and management fee ?
Probably I missed while reading this very lengthy document.

Its written in findings

\textbf{Findings implied that US equity mutual funds mangers do not
posses significant market timing skill,however there are some evidence
of `stock picking' or selection skills demonstrated by manger.}

From which part of the results you draw this result. I think one should
restrict to what one has evaluated and say that measures we have used
show that the 65 funds we have selected have ``VALUE fund investment''
is better strategy. As mentioned earlier \textbf{sample seems
non-representative as well as lopsided}. - How many large cap growth,
value, small cap growth/value funds in US funds market? What should be
sampling strategy and which sampling strategy is listed in literature in
such a case. - On the basis of alpha or any technical calculations , one
cant make statement about growth fund managers unless one studies
manager\ldots.

\hypertarget{conclusions}{%
\subsection{Conclusions}\label{conclusions}}

How have you reached at the following two conclusions?

\textbf{In perspective of new researches, more investment style
categories be included so draw inferences if any other fund category
provide understanding to investor to diversify their investment.}

\textbf{Investment in small cap funds is essential as small funds
categories performed well on alpha intercept by obtaining more funds
with positive selectivity. It is also evident form average Jensen alpha
for small growth and value funds. So it is highly recommended that
investor and mangers should consider diverse investment policy.}

\hypertarget{suggestions}{%
\subsection{Suggestions}\label{suggestions}}

\begin{itemize}
\item
  Storytelling way
\item
  Clear and precise research question
\item
  Choice of time period needs to be justified
\item
  How findings of 2005-09 data are valid today : issue of internal and
  external validity and change in nature of economy over last 15 years
\item
  Russel 1000 index for growth and value funds investments: 4 cycles
\item
  Short time period missed boom and bust cycles
\item
  Title: Exploring growth vs value funds while conclusions are like some
  breakthrough strategy to invest in value funds.
\item
  Need to update data before generalizing results
\item
  Have a precise research question(s) whether technical analysis, funds
  performance or valuation of funds
\item
  Sample size should be at least acceptable
\item
  Implications for domestic market
\end{itemize}



\end{document}
