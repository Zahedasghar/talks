\documentclass[11pt,a4paper,]{letter}
\usepackage{bera}

\usepackage{ifxetex,ifluatex}
\usepackage{fixltx2e} % provides \textsubscript
\ifnum 0\ifxetex 1\fi\ifluatex 1\fi=0 % if pdftex
  \usepackage[T1]{fontenc}
  \usepackage[utf8]{inputenc}
\else % if luatex or xelatex
  \usepackage{unicode-math}
  \defaultfontfeatures{Ligatures=TeX,Scale=MatchLowercase}
\fi
% use upquote if available, for straight quotes in verbatim environments
\IfFileExists{upquote.sty}{\usepackage{upquote}}{}
% use microtype if available
\IfFileExists{microtype.sty}{%
\usepackage[]{microtype}
\UseMicrotypeSet[protrusion]{basicmath} % disable protrusion for tt fonts
}{}
\usepackage{geometry}
\geometry{a4paper, top=2cm, bottom=2cm, left=2cm, right=2cm}
\usepackage{longtable,booktabs}
% Fix footnotes in tables (requires footnote package)
\IfFileExists{footnote.sty}{\usepackage{footnote}\makesavenoteenv{long table}}{}
\usepackage{graphicx}
\usepackage{grffile}
\makeatletter
\def\maxwidth{\ifdim\Gin@nat@width>\linewidth\linewidth\else\Gin@nat@width\fi}
\def\maxheight{\ifdim\Gin@nat@height>\textheight\textheight\else\Gin@nat@height\fi}
\makeatother
% Scale images if necessary, so that they will not overflow the page
% margins by default, and it is still possible to overwrite the defaults
% using explicit options in \includegraphics[width, height, ...]{}
\setkeys{Gin}{width=\maxwidth,height=\maxheight,keepaspectratio}
\IfFileExists{parskip.sty}{%
\usepackage{parskip}
}{% else
\setlength{\parindent}{0pt}
\setlength{\parskip}{6pt plus 2pt minus 1pt}
}
\setlength{\emergencystretch}{3em}  % prevent overfull lines
\providecommand{\tightlist}{%
  \setlength{\itemsep}{0pt}\setlength{\parskip}{0pt}}
\setcounter{secnumdepth}{0}
% Redefines (sub)paragraphs to behave more like sections
\ifx\paragraph\undefined\else
\let\oldparagraph\paragraph
\renewcommand{\paragraph}[1]{\oldparagraph{#1}\mbox{}}
\fi
\ifx\subparagraph\undefined\else
\let\oldsubparagraph\subparagraph
\renewcommand{\subparagraph}[1]{\oldsubparagraph{#1}\mbox{}}
\fi

\usepackage{color,hyperref,url,fontawesome}
\urlstyle{same}  % don't use monospace font for urls
\usepackage[absolute,overlay]{textpos}
\setlength{\TPHorizModule}{1cm}
\setlength{\TPVertModule}{1cm}

\makeatletter
\def\ps@monash{%
\begin{textblock}{4}(2,1)
\includegraphics[height=1.5cm]{_extensions/numbats/letter/monash2}
\end{textblock}%
\begin{textblock}{4}(17,1)
\includegraphics[height=1.5cm]{_extensions/numbats/letter/MBSportrait}
\end{textblock}%
\begin{textblock}{10}(1.2,26.4)
{\fontsize{9}{8}\selectfont\sffamily\color[gray]{0.4}%
\begin{tabular}{@{}l@{}}
\textbf{Zahid Asghar}, Economics, PhD\\
Professor\\
Department of Econometrics \& Business Statistics\\
Monash University, Victoria 3800, Australia.\\[0.15cm]
\faicon{envelope} zasghar@qau.edu.pk\quad\faicon{phone} +51 9064
3037\quad
\faicon{home} zahidasghar.com\\[0.1cm]
ABN: 12 377 614 012\quad CRICOS Provider Number: 00008C
\end{tabular}}
\end{textblock}%
\begin{textblock}{7}(12.7,28.2)\hfill
\includegraphics[height=0.6cm]{_extensions/numbats/letter/AACSB}~~~
\includegraphics[height=0.6cm]{_extensions/numbats/letter/EQUIS}~~~
\includegraphics[height=0.6cm]{_extensions/numbats/letter/AMBA}
\end{textblock}
\def\thepage{}}

\@ifundefined{opening}{}{%
\renewcommand*{\opening}[1]{\thispagestyle{monash}%
   {\@date\par}%
  \vspace{2\parskip}%
  {\raggedright \toname \\ \toaddress \par}%
  \vspace{2\parskip}%
  #1\par\nobreak}}
\makeatother

\def\section#1{\vspace{0.3cm}\par{\textsf{\bfseries\Large #1}}\vspace*{0.02cm}\par}
\def\subsection#1{\vspace{0.3cm}\par{\textsf{\bfseries\large #1}}\vspace*{0.02cm}\par} %}

% Date
\def\Date{\number\day}
\def\Month{\ifcase\month\or
 January\or February\or March\or April\or May\or June\or
 July\or August\or September\or October\or November\or December\fi}
\def\Year{\number\year}

% Spacing
\RequirePackage{setspace}

% Fix href problems
\def\href#1{}

% Find signature file
% Other images include full path in case user sets graphicspath
\graphicspath{{_extensions/numbats/letter/}}

\makeatletter
\makeatother
\makeatletter
\makeatother
\makeatletter
\@ifpackageloaded{caption}{}{\usepackage{caption}}
\AtBeginDocument{%
\ifdefined\contentsname
  \renewcommand*\contentsname{Table of contents}
\else
  \newcommand\contentsname{Table of contents}
\fi
\ifdefined\listfigurename
  \renewcommand*\listfigurename{List of Figures}
\else
  \newcommand\listfigurename{List of Figures}
\fi
\ifdefined\listtablename
  \renewcommand*\listtablename{List of Tables}
\else
  \newcommand\listtablename{List of Tables}
\fi
\ifdefined\figurename
  \renewcommand*\figurename{Figure}
\else
  \newcommand\figurename{Figure}
\fi
\ifdefined\tablename
  \renewcommand*\tablename{Table}
\else
  \newcommand\tablename{Table}
\fi
}
\@ifpackageloaded{float}{}{\usepackage{float}}
\floatstyle{ruled}
\@ifundefined{c@chapter}{\newfloat{codelisting}{h}{lop}}{\newfloat{codelisting}{h}{lop}[chapter]}
\floatname{codelisting}{Listing}
\newcommand*\listoflistings{\listof{codelisting}{List of Listings}}
\makeatother
\makeatletter
\@ifpackageloaded{caption}{}{\usepackage{caption}}
\@ifpackageloaded{subcaption}{}{\usepackage{subcaption}}
\makeatother
\makeatletter
\@ifpackageloaded{tcolorbox}{}{\usepackage[skins,breakable]{tcolorbox}}
\makeatother
\makeatletter
\@ifundefined{shadecolor}{\definecolor{shadecolor}{rgb}{.97, .97, .97}}
\makeatother
\makeatletter
\makeatother
\makeatletter
\makeatother

\date{\Date~\Month~\Year}

\begin{document}
\ifdefined\Shaded\renewenvironment{Shaded}{\begin{tcolorbox}[sharp corners, interior hidden, breakable, frame hidden, boxrule=0pt, borderline west={3pt}{0pt}{shadecolor}, enhanced]}{\end{tcolorbox}}\fi
\begin{letter}{Vice Chancellor\\Quaid-i-Azam University\\Islamabad}
\setstretch{1}
\vspace*{1cm}\enlargethispage*{-2cm}
\opening{Dear Sir}
\setstretch{1.4}
\begin{verbatim}
  **Subject**: *Apporval of Micro-credentials center*
\end{verbatim}

In the fourth industrial revolution or digital era, people need to
continually update their knowledge, skills and competences to fill the
gap between their education and training and the demands of a rapidly
changing labour market. Pakistan like rest of the world degree-based
university education system is finding it very challenging to equip
graduates with skills which are either becoming obsolete or yet not
known. The future of traditional higher education and routine (after
automatable) jobs is uncertain, and many potential futures exist. To
move from a future, we don't want to move to a future we want, we have
to think and practice boldly for benefiting from technologies of the
time. How to benefit from virtual learning (one of the forum unavailable
two decades back) opportunity to bridge this skill gap and prepare our
labour force for the Gig economy?

Micro-credentials are helping to bridge skill gap emerging due to fast
penetration of technological and digital technologies in the labour
market. Main purposes of micro-credentials are employability, upskilling
and reskilling, lifelong learning, adult education and inclusiveness.
The Worthy Vice Chancellor has realized the need for this and has taken
a step towards the establishment of QAU Micro-credentials centre and
appointment of Director. This is an initiative towards reskilling and
upskilling of labour force credentials irrespective of gender, age,
income and geographic location. Centre will provide opportunity for
learning as per requirements of the 3rd decade of the 21st century which
is very affordable and accessible to all.

\begin{quote}
So far, there is no national approach which demands higher education
institutions for developing micro-credentials programs which ensure
quality, relevance, and some minimum standards so that these
micro-credentials are acceptable in other parts of the world as well.
This centre will hopefully work in partnership with higher education
commission and other institutions on the subject matter.
\end{quote}

Therefore, it is requested that formal approval of the centre may be
given so that work on this national cause is started without any further
delay.
\closing{Best
wishes\\[0.2cm]\hspace*{0.5cm}\includegraphics[height=1.5cm]{sigfile.jpg}}
\ps{PS: Direct/Registrar.}
\end{letter}


\end{document}
